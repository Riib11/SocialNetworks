\documentclass{article}

%-----------------------------------------------------------------------------------------------------------------------------
%-----------------------------------------------------------------------------------------------------------------------------
% packages

% layout
\usepackage[a4paper, total={5in, 8in}]{geometry}
% \usepackage[margin=1.5in]{geometry}

% font
\usepackage{romande}
\usepackage[T1]{fontenc}
% \usepackage{gfsartemisia-euler}
% \usepackage[T1]{fontenc}

% graphics
\usepackage{graphicx}
\graphicspath{ {./images/} }

%-----------------------------------------------------------------------------------------------------------------------------
%-----------------------------------------------------------------------------------------------------------------------------
% commands

\newcommand{\todoblock}[1]{\noindent\textit{todo}: #1}
\newcommand{\todo}[1]{\textit{todo}: #1}

\renewcommand{\it}{\textit}

%-----------------------------------------------------------------------------------------------------------------------------
%-----------------------------------------------------------------------------------------------------------------------------
% header commands

\title{\sc A Network-level View \\ of Author Influence}
\author{Henry Blanchette}
\date{May 1, 2019}

%-----------------------------------------------------------------------------------------------------------------------------
%-----------------------------------------------------------------------------------------------------------------------------
% document

\begin{document}

%-----------------------------------------------------------------------------------------------------------------------------
%-----------------------------------------------------------------------------------------------------------------------------
% header

\maketitle

% \tableofcontents

%-----------------------------------------------------------------------------------------------------------------------------
%-----------------------------------------------------------------------------------------------------------------------------
\section{Introduction}
\label{sec:introduction}

The data set I used for this study is the record of papers published in a selection of computer systems conferences (categorized by Frachtenburg) and a few control, non-computer systems conferences from 2017.
The original format was the raw PDF's of all of these papers.
During the summer of 2018, Frachtenburg and his research assistants extracted features from the PDF's including author names, paper titles, and bibliographies.
The second layer of data embelleshment was matching authors with Google scholar accounts and personal survey results in order to attach author features such as email, institution, country, and gender.
Finally, a third layer of data embelleshment was matching authors and papers to Semantic Scholar (\todo{citation}) entries.
Nearly all the papers and authors (\todo{percentage}) from the original PDF's were found in Semantic Scholar.
Importantly, these Semantic Scholar identities significantly contributed to author disambiguation and retrieved the papers from previous years that were cited by the 2017 papers.

The stated purpose for my work on and analysis of this data was to look for patterns in the authors' and papers' individual-level features (from the embelleshed data) and network-level features.
I analyzed several different networks, focussing on the collaboration and citation connections between authors, papers, and conferences.
I found that the author collaboration network was particularly interesting because it displayed some significant, meaninful and separate correlations between individual-level features and network-level features.
These correlations were novel in the context of some more tradition \it{author reputation metrics}, which suggest that they measure an aspect of influence that is not well measured by the reputation metrics.

%-----------------------------------------------------------------------------------------------------------------------------
%-----------------------------------------------------------------------------------------------------------------------------
\section{Background}
\label{sec:background}

The study of social networks in general has detailed how networks can give rise to novel, emergent and important features.
These sorts of features include: diameter, connected components, bridges, distance, and centrality.
I mainly focus on centrality, but the others (especially connected components and bridges) will also play well into the later analyses as complements in explaining centrality distributions.

A node's centrality in a network can have several different flavors as will be described, but abstract, centrality is a how \it{influential} a node is on the rest of the network (where ``influential'' is in terms of whatever the edges in the network represent).
There are a few specific measures of node centrality that have proven most useful for many network contexts.
They are \it{degree}, \it{eigenvector}, \it{betweenness}, and \it{closeness} centrality.

Degree centrality is the simplest.
The \it{degree} of a node is the number of edges that include the node (note that there can also concepts of in-degree and out-degree that only count edges of a certain orientation with respect to the given node).
Then the degree centrality of a node is degined just to be the node's degree.
Often, since centralities are considered at a network-level context, degree centrality is measured differently from degree only in being normalized over the degrees of the nodes of the entire network.
Degree centrality is a measure of how prolifically connected a node is to the network, as just a raw count of how many connections it has.
This measure is the most common used in network analysis, but it is also the most local of the centrality measures because it does not consider the arrangement of the network outside of each nodes' direct neighbors.

Eigenvector centrality is more complicated.
The concept is that it weights the centrality of each node by the centralities of its neighbors, after starting from some base ranking of nodes by degree.
In this way, it is closely related to PageRank (\todo{citation}).
Formally, the eigenvector centrality of the nodes in a network is given by the vector $x$ that satisfies the equation $$ A x = \lambda x $$ where $A$ is the adjacency matrix of the network and $\lambda$ is an eigenvector with non-negative entries (\todo{citation}).
This requirement yields that there is a unique solution up to scaling. Therefor, eigenvector centrality is only meaningfully considered when normalized over the eigenvector centralities of all the nodes in the network.
Eigenvector centrality is a network-level-focussed measure because it weights heavily not just the direct neighbors of a node but also the neighbors of its neighbors and so on with diminishing effect.

Betweenness centrality measures how efficiently a node connects the network.
A given node $a$'s betweenness centrality is defined to be, over all pairs of nodes $c, d$ in the $a$'s connected component, the fraction of shortest paths between $c,d$ that include $a$.
Finally, the fraction is weighted by the size in nodes of $a$'s connected component.
Betweenness centrality is strongly connected to the concepts of connected components and bridges in a network. A connected component is a subset of the nodes of a network that have paths between any pair of them, but no paths to nodes outside the component.
Bridges are nodes that connect two would-be (non-trivial) connected components.
These bridges often have high betweenness because any path, and thus including the shortest ones, between nodes on opposite sides of the bridge must go through it.

Closeness centrality measures how close to the ``center of mass'' of the network a node is.
Closeness of a node is defined to be the connected-component-weighted reciprocal of its \textit{distance}, where the distance of a node is the mean length of the shortest paths between it and each other node in its connected component.
In a geographical way, many networks arrange themselves with a sort of planar flatness, so closeness centrality can be seen as how close a node literally is to the geographical center of the network.
Closeness is related to how involved a node is with the rest of its connected component as opposed to being involved with only a specific fringe (even if it is avidly involved which would lead to a high degree centrality).

All of these metrics are useful in the context of measuring ``influence,'' but, of course, influence can be specified more clearly in specific situations.
For example, scientific collaboration analysts of computer science fields have used degree centrality alng with correlations of node traits to explain why authors collaborate and how likely authors' with certain traits (e.g. gender, institution type) are to collaborate in the future [Ghiasi G.].
Along with degree, connections between nodes can be classified by other individual- and network-level features of the connected nodes [Ghiasi G].

More generally, the collaboration networks of scientific researhers have suggested that some specific features are important for how likely authors are to collaborate. These features include: co-authorship distance in the author collaboration network, geographical closeness, cultural closeness, and likeness of research institution type (e.g. university or indistry) [Knoke D, Yang S].
Some interesting results pointed to the conclusion that, perhaps, inter-disciplinary collaboration may be comparably important to intra-disciplinary collaboration [Knoke, Yang S].

%-----------------------------------------------------------------------------------------------------------------------------
%-----------------------------------------------------------------------------------------------------------------------------
\section{Hypotheses}
\label{sec:hypotheses}

My first inquiry was into the topical relationships of the conferences of the papers into the data.
Which conferences were most closely related to ``computer systems'' research, and how did the other distribute?
The data also contained papers from conferences that were identified as outside of computer systems in order to but the topical categorization into scale.
My hypothesis was that the typical intuitions about which computer systems conferences were most computer-systems-y would be right.
Of all people the scientists should know what kind of work they are doing, right?

My second inquiry was into the significance of typical reputation measures for authors, such as hindex and i10index, at the network level of analysis. More specifically, I predicted that reputation would correlate with the network-level sense of centrality that would be captured by some of the centrality metrics detailed in section \ref{sec:background}. The relationship between reputation and network-level node features should reveal something in regards to what the reputation metrics are and aren't good measures of.

%-----------------------------------------------------------------------------------------------------------------------------
%-----------------------------------------------------------------------------------------------------------------------------
\section{Conference Citation}
\label{sec:conference-citation}

In the conference citation network, each node represents a 2017 conference and each directed edge represents a citation by a paper in the source node's conference of a paper in the target node's conference.
Multiple with the same source and target combine into a single edge by summing weights (where weight is the number of citations that an edge represents).
These citations can go back in time however, as the conferences included in the data repeat regularly over the last couple decades.
Many citations were of conferences that were not present in the 2017 data; these were excluded from analysis.

\todoblock{conference citation network figure with labels, colors, sizes.}

\todoblock{look back at this figure and make some judgements.}

\todoblock{relate to first research question of how system-y certain conferences are.}

%-----------------------------------------------------------------------------------------------------------------------------
%-----------------------------------------------------------------------------------------------------------------------------
\section{Paper Collaboration}
\label{sec:paper-collaboration}

In the paper collaboration network, each node represents a paper from the 2017 data and each undirected edge represents an author that is shared by each of the connected node's papers.
Multiple edges between the same two nodes combine into a single edge by summing weights (where weight is the number of shared authors that an edge represents).

\begin{figure}[h!]
  \centering
  \includegraphics
    [width=1.0\textwidth]
    {paper-collaboration-network-centrality-degree.png}
  \caption{The paper collaboration network where the size and color of nodes correspond to their degree centrality. The color scale is from blue (low centrality) to white (middling centrality) to red (high centrality).}
  \label{fig:paper-collaboration-network-centrality-degree}
\end{figure}

As \ref{fig:paper-collaboration-network-centrality-degree} illustrates, this network is very clean visually.
It also gives a sense for the shape of the network.
There is a distinct, largest connected component that contains the vast majority of the highly degree-centric nodes.
\todo{number of connected components and distribution of sizes}
Outside of this there are only a few other notably-large components, and only one of which has notably-high degree-centric nodes (located in the bottom-right of figure \ref{fig:paper-collaboration-network-centrality-degree}).
\todo{name some papers in this group}

I found, however, that for the purpose of analyzing the features of authors, this network was sub-optimal.
Grouping together authors by paper obscure what are likely important author attribuets that are heterogeneous among the co-authors of a particular paper. Taking these considerations into account, I went on to create analyze the network in section \ref{sec:author-collaboration}.

%-----------------------------------------------------------------------------------------------------------------------------
%-----------------------------------------------------------------------------------------------------------------------------
\section{Author Collaboration}
\label{sec:author-collaboration}

In the author collaboration network, each node represents an author and each undirected edge represents a paper from the 2017 data that the connected authors collaborated on.
Multiple edges between the same nodes are combined into a single edge by summing weights (where the weight of an edge is the number of co-authored papers between the connected nodes' authors).

\begin{figure}[h!]
  \centering
  \includegraphics
    [width=1.0\textwidth]
    {authors-network-centrality-degree.png}
  \caption{The author collaboration network where the size and color of nodes correspond to their degree centrality. The color scale is from blue (low centrality) to white (middling centrality) to red (high centrality).}
  \label{fig:authors-network-centrality-degree}
\end{figure}

As is immediately obvious from figure \ref{fig:authors-network-centrality-degree}, there is one particular group of nodes that is skewing the degree centrality of the whole network.
This fully-connected clique of authors are those that authored the \todo{which paper is this?} paper.

\todoblock{figures for all the other centraltiy metrics}

\todoblock{distributions of centralities}

%-----------------------------------------------------------------------------------------------------------------------------
%-----------------------------------------------------------------------------------------------------------------------------
\section{Conclusions}
\label{sec:conclusions}

\subsection{Conferences' Thematic Relations to Computer Systems}

\todo{reflect on observations in the conference section}

\subsection{Reputation and Centrality}

\begin{figure}[h!]
  \centering
  \todoblock{authors features pairs matrix image}
  % \includegraphics
  %   [width=1.0\textwidth]
  %   {authors-network-centrality-degree.png}
  \caption{\todo{}}
  \label{fig:authors-feature-pairs-matrix}
\end{figure}

\todoblock{definitions of authors features}

In section \ref{sec:author-collaboration}, I listed and illustrated the network-level features of the author collaboration network.
In figure \ref{fig:authors-feature-pairs-matrix}, the network-level node centralities along with all other recorded author features are correlated against each other in pairs.
This correlations matrix yields some interesting results about how the different centrality distributions can be intepreted, but first to justify the pruning of some less-interesting correlations.

First of all, the following features did not correlate interestingly with any other features:
\begin{itemize}
\item as\_pc chair
\item as\_pc
\item as\_session\_chair
\item as\_panelist
\item as\_keynote\_speaker
\end{itemize}
This is likely because these specific measures turn out to be extremeley bi-modally and coarsly distributed as opposed to the other author features.
If an author is a panelist in 2017 they only do it once or twice and the vast majority of authors are never panelists, and likewise for as\_pc\_chair, as\_pc, etc.

Secondly, the following features are extremeley highly correlated (\todo{to what degree? look at total network pairs matrix}) with each other and correlate highly similarly with every other feature: h10index, h10index5y, i10index, i10index5y.
Since each of these is a popular reputation metric for authors, this result is unsurprising; they are all attempting to measure roughly the same thing.
I chose h10index as a representative of this group so that they other may be omitted from further correlations matrices.

The most interesting results were that of how the different centrality metrics correlated and did not correlate with the author features that I expected to be related to centrality/influence in the collaborations network.
I analyzed each centrality metric individually.

The eigenvector centrality distribution was extremeley bi-modal in a similar way to the as panelist feature, but without good reason to be so.
Eitan and I looked at some of the authors that obtained high centrality scores: \todo{examples of high eigenvector centrality nodes}.
Since the distribution of these centralities was so incomparable to all other metrics and it seemed to elevate authors that did not have a clear reason for being considered so highly influential, I decided to omit this metric from further consideration.
Eigenvector centrality can be thought of as a crude version of PageRank, which clearly has very good use cases.
For example, it is accepted as well correlated with ``relatedness'' in the context of the network of web pages connected by hyperlinks.

However, the social network of author collaborations seems to be a case where this kind of centrality was not related to the concepts of influence or reputation.
Perhaps eigenvector centrality placed too much weight on certain connections because nodes in this network tend not to have more than a few edges (\todo{node degree distribution}).
Additionally, eigenvector centrality seemed to not be commonly used in the context of social networks in general so there may be a mismatch in setup of social networks and ``reference networks'' in regards to the ideal for eigenvector centrality.


The degree centrality distribution was one of the more immediately calculable and obvious metrics.
It is also one of the usual metrics considered in social network research.
Degree centrality correlated highest with betweenness ($0.46$) and as\_author ($0.50$).
It correlated negligably ($< 0.3$) with all other author features.

The fact that degree and betweenness centralities correlate made sense because being connected to more nodes increases the chances that a node will end up on a shortest path.
However the correlation is not extremeley high (it was still $< 0.6$), so there were clearly differences in what attributes many co-authors to an author and how exactly an author chooses their co-authors as to be most efficiently connected to the rest of the network.

The fact that degree and as\_author correlate made even more sense because authoring more papers usually involves more than just exctly the same people for each paper.
Again the correlation is not extremeley high (it is still $< 0.6$), so there was a effect of authors ``in-collaborating'' with authors that they have already collaborated with during the year.
This was expected because authors are more likely to collaborate with other authors tehy already know, and one way to get to know a co-author is in fact by co-authoring with them.

The more unexpected results were the lack or correlation between degree centrality and each of npubs and hindex.
This is significant because it strongly suggested that the raw number of collaborations that an author does has little to do with either an author's reputation (hindex) or how likely they are to publish papers (npubs).
So collaborativeness is mostly orthoganol to reputation it seemed.



%-----------------------------------------------------------------------------------------------------------------------------
%-----------------------------------------------------------------------------------------------------------------------------
\section*{References}


\end{document}
